\newcommand{\R}{\mathbb{R}}
\newcommand{\N}{\mathbb{N}}
\newcommand{\Z}{\mathbb{Z}}
\newcommand{\bo}{\mathbf}
\newcommand{\it}{\mathit}
\newcommand{\mat}[1]{\begin{pmatrix}#1\end{pmatrix}}
\newcommand{\Span}[1]{\textrm{ span}\left\{#1\right\}}
\newcommand{\vp}{\vec{\varphi}}
\newcommand{\vt}{\vec{t}}
\newcommand{\ve}{\vec{e}}
\newcommand{\vn}{\vec{n}}
\newcommand{\vz}{\vec{z}}
\newcommand{\dx}{\,\mathrm{d}x\,}
\newcommand{\dy}{\,\mathrm{d}y\,}
\newcommand{\dz}{\,\mathrm{d}z\,}
\newcommand{\dV}{\,\mathrm{d}V\,}
\newcommand{\Tau}{\mathcal{T}}
\newcommand{\dof}{\textit{dof} }
\newcommand{\dofs}{\textit{dof}s }
\newcommand{\bdxd}{\left(\begin{array}{ccc}}
\newcommand{\edxd}{\end{array}\right)}
\newcommand{\Int}{\int\limits}

\documentclass[a4paper,11pt]{article}
\usepackage[a4paper, left=3cm, right=2cm, top=4cm, bottom=2cm]{geometry}

\usepackage[english]{babel}
\usepackage[T1]{fontenc}
\usepackage[latin1]{inputenc}
\usepackage{lmodern}
\usepackage{graphicx}
\usepackage{amsmath}
\usepackage{amsthm}
\usepackage{amsfonts}
\usepackage{geometry}
\usepackage{fancyhdr}
\usepackage{enumerate}
\usepackage{color}
%\usepackage{wrapfig}
\usepackage{subcaption}

\definecolor{red}{rgb}{.8,0,0}
\definecolor{black}{rgb}{0,0,0}
\definecolor{gray}{rgb}{.7,.7,.7}
\definecolor{blue}{rgb}{0,0,.8}
\definecolor{green}{rgb}{0,.8,0}


%\newcounter{thm}
%\newtheorem{stz}[thm]{Satz}
%\newtheorem{lem}[thm]{Lemma}


\begin{document}
\pagestyle{fancy}

%\begin{align*}
%I = \bdxd I_{11} & I_{12} & I_{13} \\ I_{21} & I_{22} & I_{23} \\ I_{31} & I_{32} & I_{33} \\ \edxd \\
%I_i = \left( \int_{V_i} \rho_i (r^2\delta_{\alpha\beta}-x_\alpha x_\beta) \dV \right)_{\alpha\beta} \\
%I=\sum_i m_i \bdxd y_i^2+z_i^2 & -x_i y_i &  x_i z_i \\ -y_ix_i & x_i^2+z_i^2 & -y_iz_i \\ -z_ix_i & -z_iy_i & x_i^2+y_i^2  \edxd\\
Integral exemplarisch f�r eine Diagonal und eine Nichtdiagonalkomponente
\begin{align*}
(I_i)_{11}=&\rho_i\Int_{x_i-r/2}^{x_i+r/2}\Int_{y_i-r/2}^{y_i+r/2}\Int_{z_i-r/2}^{z_i+r/2} (y-y_0)^2+(z-z_0)^2\dz\dy\dx \\
=&\rho_i\Int_{y_i-r/2}^{y_i+r/2}\Int_{z_i-r/2}^{z_i+r/2}\left[ ((y-y_0)^2+(z-z_0)^2)z \right]_{z_i-r/2}^{z_i+r/2}\dz\dy \\
=&\rho_i r\Int_{y_i-r/2}^{y_i+r/2}\Int_{z_i-r/2}^{z_i+r/2}(y-y_0)^2+(z-z_0)^2\dz\dy \\
=&\rho_i r^2 (\Int_{y_i-r/2}^{y_i+r/2}(y-y_0)^2\dy+\Int_{z_i-r/2}^{z_i+r/2}(z-z_0)^2\dz) \\
=&\rho_i r^2\left[\frac{(y-y_0)^3}{3}\right]_{y_i-r/2}^{y_i+r/2}+r^2\left[\frac{(z-z_0)^3}{3}\right]_{z_i-r/2}^{z_i+r/2} \\
=&\rho_i r^3(y_0^2-y_0(2y_i)+z_0^2-z_0(2z_i)+(r^2 /6+y_i^2+z_i^2))\\
=&y_0^2(m_i)-y_0( 2y_i m_i)+z_0^2( m_i)-z_0( 2z_i m_i)+(r^2 /6+y_i^2+z_i^2)m_i\\
(I_i)_{12}=&\rho_i\Int_{x_i-r/2}^{x_i+r/2}\Int_{y_i-r/2}^{y_i+r/2}\Int_{z_i-r/2}^{z_i+r/2} -(x-x_0)(y-y_0)\dz\dy\dx \\
=&\rho_i r\Int_{x_i-r/2}^{x_i+r/2}\Int_{y_i-r/2}^{y_i+r/2} -(x-x_0)(y-y_0)\dy\dx \\
=&\rho_i r\Int_{x_i-r/2}^{x_i+r/2}-(x-x_0)\left[ \frac{(y-y_0)^2}{2}\right]_{y_i-r/2}^{y_i+r/2}\dx \\
=&-\rho_i r\left[ \frac{(x-x_0)^2}{2}\right]_{x_i-r/2}^{x_i+r/2}\left[ \frac{(y-y_0)^2}{2}\right]_{y_i-r/2}^{y_i+r/2} \\
=&-\rho_i r^3 (x_i-x_0)(y_i-y_0)\\
=&-x_0y_0(m_i)+x_0(y_im_i)+y_0(x_im_i)-x_i y_i m_i\\
\end{align*}
Ergebniss:
\begin{align*}
I=&x_0^2\sum_im_i\bdxd 0 & 0 & 0 \\ 0 & 1 & 0\\ 0 & 0 & 1\edxd 
+ y_0^2\sum_im_i\bdxd 1 & 0 & 0 \\ 0 & 0 & 0\\ 0 & 0 & 1\edxd 
+ z_0^2\sum_im_i\bdxd 1 & 0 & 0 \\ 0 & 1 & 0\\ 0 & 0 & 0\edxd\\
-&x_0y_0\sum_im_i\bdxd 0 & 1 & 0 \\ 1 & 0 & 0 \\ 0& 0 & 0\edxd
-x_0z_0\sum_im_i\bdxd 0 & 0 & 1 \\ 0 & 0 & 0 \\1& 0 & 0\edxd
-y_0z_0\sum_im_i\bdxd 0 & 0 & 0 \\ 0 & 0 & 1 \\ 0& 1 & 0\edxd\\
+&x_0\sum_im_i\bdxd 0 & y_i & z_i \\ y_i&-2x_i &0 \\z_i & 0 & -2x_i \\  \edxd
+y_0\sum_im_i\bdxd -2y_i & x_i & 0 \\ x_i& 0 & z_i \\0 & z_i & -2y_i \\  \edxd
+z_0\sum_im_i\bdxd -2z_i & 0 & x_i \\ 0 &- 2z_i &y_i \\x_i & y_i & 0 \\  \edxd \\
+&\sum_im_i\bdxd y_i^2+z_i^2+r^2/6 & -x_iy_i & -x_iz_i \\ -x_iy_i & x_i^2+z_i^2+r^2/6 & -y_iz_i \\ -x_iz_i & -y_iz_i & x_i^2+y_i^2+r^2/6 \\ \edxd
\end{align*}
Inkrementell aktuallisiert werden also:
\begin{align*}
&\sum_im_i;\,\sum_im_ix_i;\,\sum_im_iy_i;\,\sum_im_iz_i;\,\sum_im_ix_iy_i;\,\sum_im_ix_iz_i;\,\sum_im_iy_iz_i;\\
&\sum_im_i(y_i^2+z_i^2+r^2/6);\,\sum_im_i(x_i^2+z_i^2+r^2/6);\,\sum_im_i(x_i^2+y_i^2+r^2/6)\\
\end{align*}

	
\end{document}